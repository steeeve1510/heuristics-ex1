
\section*{2. big-M proof}
Let $ N, P $ be the subsets of edges with negative and positive value, respectively; $ w_N $ and $ w_P $ the absolute values of the sums of the edge weights contained in each set, like
\begin{align*}
N &= \{(i,j) \: | \: e_{i,j} \in E \wedge d_{i,j} \leq 0 \} \\
P &= \{(i,j) \: | \: e_{i,j} \in E \wedge d_{i,j} \geq 0 \} \\
w_N &= \lvert \sum_{(i,j) \in N} d_{i,j} \rvert \\ 
w_P &= \sum_{(i,j) \in P} d_{i,j}  
\end{align*}

Assume an infeasible solution $ X $ (represented by a list of edges). We can divide $ X $ into two smaller subset of edges $ X_T $ of existing edges and $ X_F $ of non-existing edges, with weights $ w_T $ and $ w_F $ respectively. For $ w_T $ we have that $ -(w_N) < w_T < w_P $, while for every possible feasible solution $ Z $  the weight $ w_Z $ is $ -(w_N) \leq w_X \leq w_P $. Adding the weights of the infeasible solution $ w_F + w_T$ must always yield a result outside of the range for $ w_Z $. Defining the constant $ big-M $ as
$$\textit{M}= max( w_N , w_P ) * 2 + 1$$
achieves this. 

By definition of $M$,  $w_N + M$ must greater than $w_P$, and vice versa. Therefore we can conclude that every infeasible solution has a worse objective value than any feasible solution.